% Activate the following line by filling in the right side. If for example the name of the root file is Main.tex, write
% "...root = Main.tex" if the chapter file is in the same directory, and "...root = ../Main.tex" if the chapter is in a subdirectory.
 
%!TEX root = mcnamara-tanaka.tex
\section{Derived manifolds}
\hiro{Let's keep a running document here of what we're learning about Spivak's work.}

Here we summarize the portions of~\cite{spivak} and~\cite{spivak-thesis} salient to our work. The notation follows~\cite{spivak-thesis} closely, and nothing original is in this section of our paper. 


\subsection{Local models and $C^\infty$ rings}
\hiro{I might eventually remove this section; it's a non-homotopical version of the ``smooth rings'' we review in the next section.}

Let $\cR$ be a small Grothendieck site. 

\enum
	\item
		Let $U: \cR \to \Top$ be a morphism of Grothendieck sites. We say $U$ is a basis for its image if and only if, for every $R \in \cR$ and every covering of $U(R)$ in $\Top$, there is a refinement of the covering by open sets in the image of $U$. 
	\item
		Given a collection of diagrams, $\cL = \{ L: \cC_L \to \cR\}$, we say a full subcategory $\cS \subset \cR$ is closed under $\cL$-limits if any $L$ factoring through $\cS$ has a limit in $\cS$. 
	\item
		$\cS$ is closed under gluing if---whenever an open cover of $T \in \cR$ is fully contained in $\cS$, then $T$ is in $\cS$ as well.
	\item
		We say a collection of objects $G \subset \ob \cR$ generates $\cS$ if $\cS$ is the smallest full subcategory of $\cR$ containing $G$, and closed under $\cL$-limits and gluing.
\enumd

\begin{example}
Let $\cR = \cE$ denote the category of finite-dimensional Euclidean spaces. We let its morphisms be all smooth maps. The forgetful map $U: \cE \to \Top$ is a basis for its image. 

We let $\cL$ denote the collection of all functors from a finite, discrete category to $\cE$. (I.e., any functor $I \to \cE$ where $I$ is a possibly empty finite set.) 

Then $G = \{\RR\}$ generates $\cE$ under $\cL$-limits. We refer to
	\eqnn
		(\cE, U, \cL, \RR)
	\eqnd
as the Euclidean category of local models. We call $\RR$ the affine line for $\cE$.
\end{example}

Note that any functor $\cR \to \sets$ preserving $\cL$-limits is determined by what it does on any full subcategory containing a $G$ which generates $\cR$. In particular, any functor $\cE \to \sets$ is determined by what it does to $\RR$ and to its smooth endomorphisms.

\begin{defn}
A $C^\infty$-ring is a covariant functor
	\eqnn
	F: \cE \to \sets
	\eqnd
preserving $\cL$-limits. We refer to $|F| := F(\RR)$ as the underlying set of $F$. 
\end{defn}

As an example, the corepresentable functor $\hom( \RR^n, -) : \cE \to \sets$ is a $C^\infty$ ring. Its value on $\RR$ is the set of smooth functions on $\RR^n$.

\begin{remark}
Since $\RR$ is a commutative ring object in the category $\cE$, $F$ induces a commutative ring structure on $|F|$ by virtue of preserving finite limits.
\end{remark}

\begin{remark}
Other geometries, notably the geometry of affine $\ZZ$-schemes, also fit into this framework of local models.
\end{remark}


\subsection{Smooth rings}
One of the most powerful principles in algebraic geometry is that affine schemes are the same thing as commutative rings. This is somewhat tautological, but is not so tautological that its transportation into the world of smooth geometry is self-evident. (For instance, it takes care to define the appropriate tensor product for which $C^\infty(M) \tensor C^\infty (N) \cong C^\infty( M \times N)$.) At the same time, we also seek a notion of ``rings'' which is sufficiently homotopical. So the $C^\infty$ rings described in the previous sections should be thought of as the tip of a homotopical iceberg, in a way similar to how cdgas are a homotopical iceberg whose tip comprises the usual notion of commutative rings. 

Let $\fun(\man,\ssets)$ be the category of all functors from the category of manifolds to the category of simplicial sets. We call $F: \man \to \ssets$ discrete if $F(M)$ if a discrete simplicial set for all manifolds $M$.

Here are some facts we will take for granted:
\enum
	\item 
		$\fun(\man,\ssets)$ has an injective model structure. Its weak equivalences are object wise---this means $F \simeq G$ if and only if $F(M) \simeq G(M)$ for every manifold $M$. Its cofibrations are object-wise as well, meaning $F \to G$ is a cofibration if and only if $F(M) \to G(M)$ is a cofibration of simplicial sets for each $M$. This model structure is proper, and cofibrantly generated.
	\item
		Let $S$ be a simplicial set. We let $\un{S}: \man \to \ssets$ denote the constant functor. This makes $\fun(\man,\ssets)$ tensored over simplicial sets.
	\item
		All discrete $F$ are fibrant. In particular, if $M$ is a manifold, the functor
			\eqnn
			H_M := \man(M, -)
			\eqnd
		is fibrant.
\enumd

\begin{defn}
The model category of smooth rings is the localization of $\fun(\man,\ssets)$ along $\Psi$.
\end{defn}

We define $\Psi$ now.
Consider a diagram
	\eqnn
		\xymatrix{
			 & P \ar[d]^a \\
			M \ar[r]^s & N
		}
	\eqnd
where $s$ is a submersion. Then the pullback $Q \cong M \times_N P$ exists in the category of smooth manifolds. On the other hand, one could also take the homotopy pushout
	\eqnn
		\xymatrix{
			H_N \ar[r]^{s^*} \ar[d]_{a^*} & H_M \ar[d] \\
			H_P \ar[r] & G
		}
	\eqnd
in the model category $\fun(\man,\ssets)$. We let
	\eqnn
		\psi_{s,a} : G \to H_Q
	\eqnd 
denote the induced map. Note that localizing with respect to these $\psi_{s,a}$ means that the assignment ``corepresenting functor" would preserve the one good operation there is in smooth manifolds: pulling back along submersions. That is, pullbacks obtained from submersions will be sent to homotopy pushouts.

Finally, note that the 0-manifold $pt \cong \RR^0$ is terminal in $\man$. We'd like this to be reflected in the category of smooth rings. (This reflects as principle from algebraic geometry: if $\spec k$ is terminal in $k$-schemes, we'd like the structure ring $k$ to be initial in $k$-algebras.) However, in $\fun(\man,\ssets)$, the initial object is not $H_{pt}$, but rather the functor sending $M \mapsto \emptyset$. So we would like to further localize with respect to the unique map
	\eqnn
	 (M \mapsto \emptyset) \to H_{pt}.
	\eqnd 

\begin{defn}
We let $\Psi$ denote the collection of all morphisms of the form $\psi_{s,a}$ (for $s$ a submersion and $a$ a smooth map) together with the unique map $(M \mapsto \emptyset) \to H_{pt}$.
\end{defn}

\hiro{
It would be nice to just think of this as an $\infty$-category from the outset, without having to resort to model categories, and then defining an $\infty$-category. But so it is.
}


\begin{example}[Example 2.1.10 in ~\cite{spivak-thesis}.]
If $M, N$ are smooth manifolds, the object-wise coproduct $H_M \coprod H_N$ is in $\fun(\man,\ssets)$. However, it is not a fibrant object in the model category of smooth rings. For instance, apply it to the pullback diagram realizing $M \times_{pt} N \cong M \times N$. However, the map
	\eqnn
		H_M \coprod H_N \to H_{M \times N}
	\eqnd
is a fibrant replacement map. 
\end{example}

\begin{example}[The fat point, 2.1.13 in~\cite{spivak-thesis}.]
Let us compute the homotopy pushout
	\eqnn
		\xymatrix{
		H_\RR \ar[r] \ar[d] & H_{pt} \ar[d] \\
		H_{pt} \ar[r] & G
		}
	\eqnd
where the arrows $H_\RR \to H_{pt}$ are induced by the inclusion of the origin in $\RR$. As usual, to compute this homotopy pushout, we just replace either the top or lefthand arrow by a cofibration. (The localization is left proper because $\fun(\man,\ssets)$ is; see~\cite{hirschhorn} 4.1.1.) Consider the composite map
	\eqnn
		g: \RR \to pt \xra{0} \RR.
	\eqnd
Then we have a functor $C: \man \to \ssets$ whose only non-degenerate simplices are in dimensions 1 and 0, given by
	\eqnn
	d_0, d_1 : H_\RR \to H_\RR
	\eqnd
where $d_0$ is the identity and $d_1$ is $g^*$. Obviously, the inclusion $H_\RR \to C$ is a cofibration because it is a levelwise injection for any manifold $M$. One can check straightforwardly that $H_{pt}(M) \simeq C(M)$ for every manifold $M$. Hence the homotopy pushout $G$ can be computed as the honest pushout of the diagram
	\eqnn
		\xymatrix{
		H_\RR \ar[r] \ar[d] & C \ar[d] \\
		H_{pt} \ar[r] & G.
		}
	\eqnd
Thus $G$ is given by the functor $\man \to \ssets$ whose non-degenerate bit we represent by
	\eqnn
	d_0, d_1: H_\RR \to H_{pt}.
	\eqnd
Here, both $d_i$ are induced by the inclusion $0: pt \to \RR$. 

\hiro{
Added after phone discussion.
Note that this sheaf of simplicial sets is not fibrant. Even when evaluated on $\RR$, or on $pt$, it is not a Kan complex. 
What we can try to prove is that, on each manifold, the map
	\eqnn
		G
		\to
		Implicit-stuff
	\eqnd
is left/right anodyne. Then by 4.1.1.3 of HTT, this map is initial/final, hence an homotopy equivalence in the Quillen/Kan sense.
Aside from the definition (4.1.1.1 of HTT) of being final, Joyal's characterization (Theorem 4.1.3.1 of HTT) may also be helpful.
}
\end{example}



\begin{remark}
We remind the reader of what localization of $\cC$ to $\cC[\Psi^{-1}]$ does, at least at a superficial level. Heuristically, it turns any morphism in $\Psi \subset \cC$ into an equivalence in the localization. In terms of model categories:
\enum
	\item
		$\cC$ and $\cC[\Psi^{-1}]$  are the same simplicial category.
	\item
		The cofibrations of $\cC[\Psi^{-1}]$ are the cofibrations of $\cC$.
	\item
		A fibrant object $X$ of $\cC[\Psi^{-1}]$ is one which is both fibrant in $\cC$, and $\Psi$-local. That is, for any $\psi : A \to B \in \Psi$, the map
			\eqnn
			\map(B, X) \to \map(A,X)
			\eqnd
		is a weak equivalence.
	\item
		The weak equivalences in $\cC[\Psi^{-1}]$ are those $f: X \to Y$ such that 
			\eqnn
			\map(Y,-) \to \map(X,-)
			\eqnd
		is a weak equivalence whenever $-$ is fibrant in $\cC[\Psi^{-1}]$.
\enumd
\end{remark}


\subsection{Sheaves, local sheaves, derived manifolds}
Let $X$ be a topological space. Let $\cF$ be a sheaf of $C^\infty$ rings. This means that $\cF$ is a contravariant functor from $\open(X)$ to $\fun(\man,\ssets)$, satisfying descent for hypercovers.

\subsubsection{Local sheaves}
\begin{defn}
We say that $\cF$ is local if, for every open cover $\{U_\alpha$ of $M$, the natural map of sheaves of sets
	\eqnn
		\pi_0 
		\left(
			\coprod_\alpha \cF(-,M_\alpha) 
		\right)
		\to
		\pi_0 \cF(-,M)
	\eqnd
is a surjection. 
\end{defn}

\begin{defn}
If $\cF$ is local, we say that the pair $(X,\cF)$ is a local smooth-ringed space.
\end{defn}

\begin{defn}
If $\cF$ and $\cG$ are sheaves, we say that $\phi: \cF \to \cG$ is a local morphism of sheaves if and only if: For any open $U \subset M$, the natural diagram
	\eqnn
		\xymatrix{
			\pi_0\cF(-,U) \ar[r]^\phi \ar[d]^{res} 
				& \pi_0 \cF( - , U) \ar[d]^{res} \\
			\pi_0 \cF(-, M) \ar[r]^\phi & \pi_0 \cF(-,M)
		}
	\eqnd
exhibits $\pi_0 \cF( - , U)$ as a pullback of sheaves of sets.

We let
	\eqnn
		\map_{\loc}(\cF,\cG) \subset \map(\cF,\cG)
	\eqnd
denote the full simplicial set spanned by local morphisms. That is, every simplex on the righthand side has vertices given by local morphisms.
\end{defn}

\begin{defn}
If $(X,\cO_X)$ and $(Y,\cO_Y)$ are local smooth-ringed spaces, we let
	\eqnn
		\map(X,Y) := \coprod_{\phi: X \to Y} \map_{\loc} (\phi^*\cO_Y, \cO_X).
	\eqnd
where $\phi$ runs over all continuous maps.
\end{defn}

The following gives some intuition for why we insist on the adjective local:

\begin{theorem}[Theorem~3.3.6 of~\cite{spivak-thesis}.]
Let $\cF$ be a sheaf of smooth ring son $X$. The following are equivalent:
\enum
	\item
		$\cF$ is local.
	\item
		For every open cover $U_\bullet \to M$ of a smooth manifold $M$, the natural map
			\eqnn
				\hocolim(\cF(-,U_\bullet)) \to \cF(-,M)
			\eqnd
		is a weak equivalence of simplicial sheaves on $X$. Here, the hocolim is over the Cech nerve of the cover.
	\item
		The same holds above if each element of the open cover $U_\bullet$ is by Euclidean spaces.
	\item
		If $f: \cF \to \cG$ is a local morphism of sheaves of smooth rings, and if $f$ induces a weak equivalence of sheaves
			\eqnn
				\cF(-,\RR) \simeq \cG(-,\RR)
			\eqnd
		then $f$ is a weak equivalence of sheaves of local smooth rings. (This is Corollary~3.3.7 of~\cite{spivak-thesis}.)
\enumd
\end{theorem}

\subsubsection{Derived manifolds}
We will show how to compute homotopy pullbacks of $C^\infty$-ringed spaces momentarily. But for now:

\begin{defn}
Let $f: \RR^n \to \RR^m$ be a smooth map. Then the $C^\infty$-ringed space given by the homotopy fiber product
	\eqnn
	\xymatrix{
		\cU \ar[r] \ar[d] & \RR^0 \ar[d]_0 \\
		\RR^n \ar[r]^f & \RR^m
	}
	\eqnd
which we write $\cU = (U, \cO_U)$, is called a principal derived manifold. 
\end{defn}

\begin{defn}
Any Hausdorff smooth-ringed space $(X,\cO_X)$ is called a derived manifold if it can be covered by countably many principal derived manifolds.
\end{defn}

In other words, derived manifolds are locally modeled on zero locuses of functions. The salient point here is that the homotopy fiber product is what produces a robust interpretation of the zero locus---one that is independent of perturbations, and which can shrug off non-transversality.

\begin{defn}
The category of derived manifolds is the full simplicial subcategory of the category of locally smooth-ringed spaces.
\end{defn}

Here are some basic properties to get the reader acclimated:

\begin{theorem}
\enum
	\item
		Any smooth manifold $M$ defines a structure sheaf $\cO_M$ which sends any manifold $N$ to the set of smooth functions from $M$ to $N$. This is a locally smooth-ringed space. (Proposition~3.2.2 of~\cite{spivak-thesis}.)
	\item
		The inclusion of smooth manifolds (with a discrete set of smooth maps) into the simplicial category of locally ringed spaces is fully faithful. (Proposition~3.2.3 of~\cite{spivak-thesis}.) 
	\item
		If $(X,\cO_X)$ is a locally smooth-ringed space, then for any smooth manifold $M$, the simplicial set of (local) maps from $(X,\cO_X)$ to $(M,\cO_M)$ is homotopy equivalent to $\cO_X(X,M)$. In other words, the global sections of $\cO_X$, evaluated on the manifold $M$, recovers the space of maps from $X$ to $M$. (Theorem~3.3.3 of~\cite{spivak-thesis}.) 
\enumd
\end{theorem}

\subsubsection{Computing with derived manifolds}
First we discuss how to compute fiber products. The algorithm is: Take the fiber product in the usual category of topological spaces, then take the homotopy pushout in the category of sheaves of simplicial rings.






\subsection{Simplicial commutative algebras}
\hiro{Section 3.5 of~\cite{spivak-thesis}.}

Smooth-ringed spaces define derived manifolds, but they rarely allow us to compute. Thankfully, replacing a smooth ring with its underlying simplicial commutative $\RR$-algebra preserves many homotopy colimits. This buys us mileage in local computations: While taking global sections is a limit, taking stalks is a colimit.

For instance, we have:

\begin{lemma}[Lemma~3.5.7 of~\cite{spivak-thesis}]
Let $\cF$ be a sheaf of smooth rings on $X$.

\end{lemma}




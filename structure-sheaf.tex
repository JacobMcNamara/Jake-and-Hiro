%!TEX root = mcnamara-tanaka.tex

\section{The structure sheaf}

Consider the simplest case of an implicit atlas $\cA$ with a single global chart, given by a smooth manifold $Y$, a smooth function $s: Y \to E$ into a finite dimensional vector space $E$, and the zero set $X = s^{-1}(0)$. Since the following diagram should be a pullback,
\[ \xymatrix{X \ar[r] \ar[d] & Y \ar[d]^-{s} \\ {*} \ar[r]_-{0} & E}\]
we would like to have that the $C^\oo$-ring $\cO(X)$ is the homotopy tensor product, $\cO(X) = \RR \tensor_{\cO(E)} \cO(Y)$.

\hiro{Great simple example, and nice motivation. This could be cast as "Example x.y.z" after we finish writing a definition we know to be correct.}

\subsection{The cospan nerve of a category}

\begin{definition}\label{cospan-nerve}
Let $\cC$ be a category. We construct a simplicial set $\CoSpan^\bullet(\cC)$, the {\bf cospan nerve} of $\cC$, as follows. Let $\cP_+(n)$ denote the poset of nonempty subsets of the set $[n] = \{0, \dots, n\}$. We define
\[ \CoSpan^n(\cC) = \Fun(\cP_+(n), \cC), \]
the set of functors from $\cP_+(n)$ to $\cC$. Given a map $f: [n] \to [m]$, we get an induced map $\CoSpan^m(\cC) \to \CoSpan^n(\cC)$ given by pulling back along $f$, and so $\CoSpan^\bullet(\cC)$ is a simplicial set.
\end{definition}

\begin{prop}\label{cospans-are-kan}
Let $\cC$ be a category with weak pushouts. Then $\CoSpan^\bullet(\cC)$ is a Kan complex.
\end{prop}

\begin{proof}
\jake{[To be completed.]}
\end{proof}

\begin{speculation}
Suppose $\cC$ is a category with weak pushouts and a weak initial object. Then $\CoSpan^\bullet(\cC)$ is contractible.
\end{speculation}

\subsection{The local $\Ex$ functor}

\jake{Reference is Goerss-Jardine, III.4}

We construct a version of the functor $\Ex: \ssets \to \ssets$ in the slice category $\ssets_{/S}$ of simplicial sets over $S$. First, we recall the relevant properties of the $\Ex$ functor. First, we have a natural inclusion $\eta: X \to \Ex(X)$ for any simplicial set $X$, which is an acyclic cofibration. Further, we have that the colimit $\Ex^\oo(X) = \lim_{n \to \oo} \Ex^n(X)$, where $\Ex^n$ denotes $n$-fold application of $\Ex$, is a fibrant replacement functor in the Quillen model structure on simplicial sets.

\begin{definition}\label{local-Ex}
Let $p: T \to S$ be a map of simplicial sets. We have a natural diagram
\[\xymatrix{
T \ar[r]^-{\eta} \ar[d]_-{p} & \Ex(T) \ar[d]^-{\Ex(p)} \\
S \ar[r]_-{\eta} & \Ex(S)
}\]
We define $\Ex_S(T) = S {}_{\eta}{\times}_{\Ex(p)} \Ex(T)$. The above diagram induces a natural inclusion $\eta_S:T \to \Ex_S(T)$ over $S$.
\end{definition}

\jake{This is the same as taking the simplicial set whose simplices are cospans in $T$, such that all the arrows pointing the ``wrong way" cover identity maps in $S$.}

\begin{prop}\label{local-Ex-weak-equivalence}
The natural map $\eta_S: T \to \Ex_S(T)$ is an acyclic cofibration in the contravariant model structure \jake{[reference is H.T.T. Remark 2.1.4.12]}.
\end{prop}

\begin{proof}
That $\eta_S$ is a cofibration is immediate from the definition, since cofibrations are just monomorphisms of simplicial sets, and $\eta$ is a monomorphism. To show that it is a weak equivalence, we must show that the induced map
\[ S \coprod_T T^\rhd \to S \coprod_{\Ex_S(T)} \Ex_S(T)^\rhd, \]
is a categorical equivalence. \jake{[Need to complete proof; use facts about $T \to \Ex(T)$.]}
\end{proof}

\begin{remark}
Let $\cX = (Y, E, f)$ be a thickening. There is a functor $\TH_\cX$ from the category of smooth vector bundles over $Y$ and smooth vector bundle inclusions to the under-category $\cT_{\cX/}$, given on objects by
\[ \TH_\cX: \left(\pi: F \to Y\right) \mapsto \left(F, \pi^* F \oplus \pi^* E, \rho_F \oplus \pi^*(f) \right),\]
where $\rho_F$ is the canonical section of $\pi^* F \to F$.
\end{remark}

\begin{prop}\label{lifting-21-horns}
Let $S$ be the nerve of the category of Hausdorff spaces and open embeddings, and let $T$ be the nerve of the category $\cT$ defined below in Definition \ref{thickening-charts}. Let $p: T \to S$ be the nerve of the functor given by sending a thickening chart $(Y, E, f)$ to the zero locus $f^{-1}(0)$. Suppose we have a diagram of simplicial sets,
\[\xymatrix{
\Lambda^2_1 \ar@{^{(}->}[d] \ar[r] & \Ex_S(T) \ar[d] \\
\Delta^2 \ar[r] \ar@{-->}[ur] & S
}\]
Then there exists a map from $\Delta^2$ to $\Ex_S(T)$, denoted by the dotted arrow, which makes the whole diagram commute.
\end{prop}

\begin{proof}

More explicitly, the data we are given is a sequence of Hausdorff spaces $X_0, X_1, X_2$, together with open embeddings $\iota_{\alpha \beta}: X_\alpha \to X_\beta$ for $\alpha \leq \beta$ such that $\iota_{\gamma \delta} \circ \iota_{\beta \gamma} = \iota_{\beta \delta}$ and $\iota_{\beta \beta} = \id_{X_\beta}$, together with a diagram of thickenings $\cX_I = (Y_I, E_I, f_I)$, and maps $\cF_{IJ} = (i_{IJ}, j_{IJ}): \cX_I \to \cX_J$,
\[\xymatrix{
\cX_{\{0\}} \ar[dr]_-{\cF_{\{0\}\{0,1\}}} & & & & \cX_{\{2\}} \ar[dl]^-{\cF_{\{2\}\{1,2\}}} \\
& \cX_{\{0, 1\}} & & \cX_{\{1,2\}} & \\
& & \cX_{\{1\}} \ar[ul]^-{\cF_{\{1\}\{0,1\}}} \ar[ur]_-{\cF_{\{1\}\{1,2\}}} & &
}\]
We have further that $f_I^{-1}(0) = X_\alpha$ whenever $\beta$ is the greatest element of $I$, and $\rest{i_{I J}}{X_\alpha} = \iota_{\alpha \beta}$, where $\alpha$ is the greatest element of $I$ and $\beta$ is the greatest element of $J$.

Now, we must construct thickenings $\cX_{\{0,1,2\}}$ and $\cX_{\{0,2\}}$ of $X_n$ fitting into the diagram as
\[\xymatrix{
\cX_{\{0\}} \ar[ddr]_-{\cF_{\{0\}\{0,1\}}} \ar[rr]_-{\cF_{\{0\}\{0,2\}}} & & \cX_{\{0, 2\}} \ar[d]^-{\cF_{\{0, 2\}\{0,1,2\}}} & & \cX_{\{2\}} \ar[ddl]^-{\cF_{\{2\}\{1,2\}}} \ar[ll]^-{\cF_{\{2\}\{0,2\}}} \\
& & \cX_{\{0,1,2\}} & & \\
& \cX_{\{0, 1\}} \ar[ur]^-{\cF_{\{0, 1\}\{0,1,2\}}} & & \cX_{\{1,2\}} \ar[ul]^-{\cF_{\{1, 2\}\{0,1,2\}}} & \\
& & \cX_{\{1\}} \ar[ul]^-{\cF_{\{1\}\{0,1\}}} \ar[ur]_-{\cF_{\{1\}\{1,2\}}} & &
}\]
Actually, it suffices to find $\cX_{\{0, 1, 2\}}$ and the maps $\cF_{\{0,1\}\{0,1,2\}}, \cF_{\{1, 2\}\{0,1,2\}}$, since we may take $\cX_{\{0, 2\}} = \cX_{\{0, 1, 2\}}$ and $\cF_{\{0,2\}\{0,1,2\}} = \id_{\cX_{\{0, 1, 2\}}}$, defining everything else by the apparent compositions.

Now to construct $\cX_{\{0,1,2\}}$, we first choose an open subset $i: U \hookrightarrow Y_{\{1,2\}}$ such that $U \cap X_2 = X_1$, and consider the thickening
\[ \cY = \rest{\cX_{\{1,2\}}}{U} = (U, \rest{E_{\{1,2\}}}{U}, \rest{E_{\{1,2\}}}{U}). \]
Now, we have a span of thickenings $\cX_{\{0,1\}} \leftarrow \cX_{\{1\}} \to \cY$, covering the identity of $X_1$. Thus, by Lemma \ref{vector-bundle-diagrams}, we may find vector bundles $F, F'$ over $Y_{\{1\}}$ such that the above span is just $\TH_{\cX_{\{1\}}}(F) \leftarrow \cX_{\{1\}} \to \TH_{\cX_{\{1\}}}(F')$. Let $\cZ = \TH_{\cX_{\{1\}}}(F \oplus F')$; we thus have a diagram of thickenings of $X_1$
\[\xymatrix{
& \cZ & \\
\cX_{\{0, 1\}} \ar[ur] & & \cY \ar[ul] \\
& \cX_{\{1\}} \ar[ul] \ar[ur] &
}\]
Now, consider the map $\cY \to \cZ$. Again by Lemma \ref{vector-bundle-diagrams}, we have that there is a vector bundle $H$ over $U$ such that $\cZ = \TH_\cY(H)$. Choose an embedding $j: H \hookrightarrow \underline{\RR^k}$ into a trivial vector bundle over $U$; this is always possible. Now, set $\cX_{\{0,1,2\}} = \TH_{\cX_{\{1,2\}}}(\underline{\RR^k})$. There is an evident map $\cZ \to \cX_{\{0,1,2\}}$, given by
\[ (i \times \id_{\RR^k}) \circ j : H \to Y_{\{1,2\}} \times \RR^k \]
on manifolds, and with the obvious formula on vector bundles. There is also an evident map $\cF_{\{1,2\}\{0,1,2\}}: \cX_{\{1, 2\}} \to \cX_{\{0,1,2\}}$. Defining $\cF_{\{0,1\}\{0,1,2\}}$ as the composition of the maps $\cX_{\{0,1\}} \to \cZ \to \cX_{\{0,1,2\}}$ completes the diagram, and thus we have the desired lifting property.


\end{proof}


\begin{prop}\label{presheaf-of-implicit-structures}
\jake{This says that we have a simplicial presheaf of thickening charts over the category of Hausdorff spaces.} Let $S$ be the nerve of the category of Hausdorff spaces and open embeddings, and let $T$ be the nerve of the category $\cT$ defined below in Definition \ref{thickening-charts}. Let $p: T \to S$ be the nerve of the functor given by sending a thickening chart $(Y, E, f)$ to the zero locus $f^{-1}(0)$. Then the map $\Ex_S(T) \to S$ is a right fibration.
\end{prop}

\begin{lemma}\label{vector-bundle-diagrams}
Let $D: \cD \to \cT_{\cX /}$ be any diagram, where $\cD$ is a poset. Suppose that, for all $I \in \cD$, we have that the map $\cX \to D(I)$ is the identity map on the zero sets. Then $D$ must factor through $\TH_\cX$, up to natural isomorphism.
\end{lemma}

\begin{proof}
\jake{[To be done.]}
\end{proof}


\begin{proof}[Proof of Proposition \ref{presheaf-of-implicit-structures}]

Suppose we are given a simplex $\Delta^n \to S$, together with a lift of $\Lambda^n_\alpha \subset \Delta^n$ to a map $\Lambda^n_\alpha \to \Ex_S(T)$. We need to show that there exists a simplex $\Delta^n \to \Ex_S(T)$ compatible with the given data as long as $0 < \alpha \leq n$. We will induct on the poset $\{ \mathrm{RH} \leq \mathrm{IH} \}$, where $\mathrm{RH}$ denotes right horns, and $\mathrm{IH}$ denotes inner horns. That is, at each step in the proof, if we make an argument that depends on casework for right horns and inner horns separately, in the case of inner horns, we will allow ourselves to assume that the proposition is already proven for right horns.

More explicitly, we are given Hausdorff spaces $X_0, \dots, X_n$, together with embeddings $\iota_{\beta \gamma}: X_\beta \hookrightarrow X_\gamma$ when $\beta \leq \gamma$ such that $\iota_{\gamma \delta} \circ \iota_{\beta \gamma} = \iota_{\beta \delta}$ and $\iota_{\beta \beta} = \id_{X_\beta}$. WLOG, we may assume that $\iota_{\beta \gamma}$ is the inclusion of an open subset. We further have thickening charts $\cX_I = (Y_I, E_I, f_I)$ for each nonempty subset $I \subset \{0, \dots, n\}$ such that $I$ does not contain $\{0, \dots, \widehat{\alpha}, \dots, n\}$. We have that $X_\beta = f_I^{-1}(0)$ whenever $\beta$ is the greatest element of $I$. We further have morphisms
\[ \cF_{IJ} = (i_{I J}, j_{I J}): \cX_I \to \cX_J, \]
when $I \subset J$, such that $\cF_{JK} \circ \cF_{IJ} = \cF_{IK}$ and $\cF_{II} = \id_{\cX_I}$. Further, if $\beta$ is the greatest element of $I$ and $\gamma$ is the greatest element of $J$, then $\rest{i_{I J}}{X_\beta} = \iota_{\beta \gamma}$.

\begin{lemma}\label{existence-of-good-thickenings}
In the above context, there exist thickenings $\wt \cX_I = (\wt Y_I, \wt E_I, \wt f_I)$ for each nonempty proper subset $I \subset \{0, \dots, n\}$ such that $\alpha \in I$, together with morphisms
\[ \wh \cF_I = (\wh i_I, \wh j_I): \cX_I \to \wt \cX_I, \quad \wt \cF_{IJ} = (\wt i_{IJ}, \wt j_{IJ}) : \wt \cX_I \to \wt \cX_J, \]
such that $\wt \cF_{JK} \circ \wt \cF_{IJ} = \wt \cF_{IK}$, $\wt \cF_{II} = \id_{\wt \cX_I}$, and $\wh \cF_J \circ \cF_{IJ} = \wt \cF_{IJ} \circ \wh \cF_I$. We will further require that $\wt f_I^{-1}(0) = X_n$ and $\rest{\wt i_{IJ}}{X_n} = \id_{X_n}$, and that if $\beta$ is the greatest element of $I$, then $\rest{\wh i_I}{X_\beta} = \iota_{\beta n}$.
\end{lemma}

\begin{proof}[Proof of Lemma \ref{existence-of-good-thickenings}]

First, suppose $I \neq \{0, \dots, n-1\}$. In this case, we have that $\{0, \dots, \wh \alpha, \dots, n\} \not\subset I \cup \{n\}$, and so we may define
\[ \wt \cX_I = \cX_{I \cup \{n\}}, \quad \wh \cF_I = \cF_{I (I \cup \{n\})}, \quad \wt \cF_{IJ} = \cF_{(I \cup \{n\})(\{J \cup \{n\})}. \]
Now, if $\alpha = n$, then we are done. Thus, assume $\alpha < n$, and further assume by induction that the map $\Ex_S(T) \to T$ has the right lighting property with respect to all right horns. Consider the lifting problem for the right horn given by the assignment
\[ I \mapsto \cX_{I \cup \{\alpha\}}, \quad \left(I \subset J \right) \mapsto \cF_{(I \cup \{ \alpha \})(J \cup \{ \alpha \})}, \]
defined for nonempty $I \subset \{0, \dots, \wh \alpha, \dots, n\}$, such that $I \not\supset \{0, \dots, \wh \alpha, \dots, n-1\}$. This horn in $\Ex_S(T)$ covers the $(n-1)$-simplex in $S$ given by
\[\xymatrix{
X_\alpha \ar[r]^-{\id_{X_\alpha}} & X_\alpha \ar[r]^-{\id_{X_\alpha}} & X_\alpha \ar[r]^-{\iota_{\alpha(\alpha+1)}} & X_{\alpha + 1} \ar[r]^-{\iota_{(\alpha+1)(\alpha+2)}} & \cdots \ar[r]^-{\iota_{(n-1)n}} & X_n
}\]
with $\alpha$ copies of $X_\alpha$. Thus, by the lifting property, we may in particular find a thickening $\cZ$ of $X_n$ and morphisms $\cG_I: \wt \cX_I \to \cZ$ for each proper nonempty subset $I \subset \{0, \dots, n\}$ such that $\alpha \in I$ and $I \neq \{0, \dots, n-1\}$, such that $\cG_J \circ \wt \cF_{IJ} = \cG_I$.

\jake{[Need to finish.]}

\end{proof}

With Lemma \ref{existence-of-good-thickenings} in hand, we proceed to solve the lifting problem. By Lemma \ref{vector-bundle-diagrams}, we may factor the functor
\[ I \mapsto \left( \wt \cF_{\{\alpha\} I}: \wt \cX_{\{\alpha\}} \to \wt \cX_I \right), \quad \left( I \subset J \right) \mapsto \cF_{I J}, \]
through the category of vector bundles over $Y_{\{\alpha\}}$, up to natural isomorphism. Thus, we may assume WLOG that we have vector bundles $\pi_I: F_I \to Y_{\{\alpha\}}$ and compatible inclusions $g_{I J} : F_I \to F_J$ of vector bundles over $Y_{\{\alpha\}}$, such that
\[ \wt \cX_I = \TH_{\cX_{\{\alpha\}}}(F_I), \quad  \wt \cF_{IJ} = \TH_{\cX_{\{\alpha\}}}(g_{IJ}). \]
Now, let $F_{\{0, \dots, n\}}$ be the colimit of $F_I$ over $I$ in the category of vector bundles and vector bundle maps over $Y_{\{\alpha\}}$, and let $g_{I\{0, \dots, n\}}: F_I \to F_{\{0, \dots, n\}}$ be the natural inclusion. Set
\[ \cX_{\{0, \dots, n\}} = \TH_{\cX_{\{\alpha\}}}(F_{\{0, \dots, n\}}), \quad \cF_{I \{0, \dots, n\}} = \TH_{\cX_{\{\alpha\}}}(g_{I \{0, \dots, n\}}) \circ \wh \cF_{I}. \]
Thus, to fully solve the lifting problem, it remains to define $\cX_{\{0, \dots, \wh \alpha, \dots, n\}}$ and the morphisms to and from it. If $\alpha \neq n$, then simply set
\[ \cX_{\{0, \dots, \wh \alpha, \dots, n\}} = \cX_{\{0, \dots, n\}}, \quad \cF_{\{0, \dots, \wh \alpha, \dots, n\} \{0, \dots, n\}} = \id_{\cX_{\{0, \dots, n\}}},\]
and
\[\cF_{I \{0, \dots, \wh \alpha, \dots, n\}} = \cF_{I' \{0, \dots, n\}} \circ \cF_{I I'},\]
where $I' = I \cup \{\alpha\}$. Otherwise, if $\alpha = n$, then choose an open subset $U \subset F_{\{0, \dots, n\}}$ such that $U \cap X_n = X_{n-1}$, which we may do since $X_{n-1}$ is open in $X_n$, and set
\[ \cX_{\{0, \dots, \wh \alpha, \dots, n\}} = \rest{\cX_{\{0, \dots, n\}}}{U}, \quad \cF_{\{0, \dots, \wh \alpha, \dots, n\} \{0, \dots, n\}} = \rest{\id_{\cX_{\{0, \dots, n\}}}}{U},\]
and
\[\cF_{I \{0, \dots, \wh \alpha, \dots, n\}} = \rest{(\cF_{I' \{0, \dots, n\}} \circ \cF_{I I'})}{V_I},\]
where $V_I$ is an open subset of $Y_I$ that lands within $U$ under the above map. Thus, we have solved the lifting problem, and $\Ex_S(T) \to S$ is a right fibration, as claimed.

\end{proof}

\subsection{Category of thickenings}

\begin{definition}\label{thickening-charts}
A {\bf thickening chart}  $\cU = (Y, E, f)$ consists of a smooth manifold $Y$, together with a smooth vector bundle $E$ over $Y$ and a smooth section $f: Y \to E$. Given two thickening charts $\cU_i = (Y_i, E_i, f_i)$ for $i = 0, 1$, a morphism $\cU_0 \to \cU_1$ is a smooth embedding $i: Y_0 \hookrightarrow Y_1$ together with an inclusion of vector bundles $j: E_0 \hookrightarrow i^* E_1$ such that $j \circ f_0 = i^*(f_1)$, and such that the induced diagram
\[\xymatrix{
		Y_0 \ar[d]_-{f_0} \ar@{^{(}->}[r]^-{i} & Y_1 \ar[d]^-{f_1} \\
		E_0 \ar@{^{(}->}[r]_-{\widetilde{\jmath}} & E_1}\]
is a transverse pullback of smooth manifolds. We let $\cT$ denote the category whose objects are thickening charts, and whose morphisms are \jake{germs of} morphisms of thickening charts.
\end{definition}

\begin{wild-speculation}\label{model-category}
Consider the category whose objects are thickening charts, and whose morphisms are the same as morphisms of thickening charts but without the condition that $i$ and $\widetilde{\jmath}$ above be embeddings or the transverse pullback condition. I think there should be a model structure on this, where we have
\begin{itemize}

\item Cofibrations are maps with $i$ and $\widetilde{\jmath}$ embeddings.

\item Weak equivalences are maps that are homeomorphisms on the zero set, and which are quasi-isomorphisms on the tangent complex \jake{[to be defined]} around the zero set.

\item Fibrations should have at least that the map $Y_0 \to Y_1$ be a submersion, plus probably the homotopy lifting property.

\end{itemize}
I think I can show that the acyclic cofibrations are just the things I'm calling morphisms of thickening charts above, with the extra condition that the zero sets be the same.
\end{wild-speculation}

\begin{remark}
\jake{There are two functors out of $\cT$ that should be very interesting. The first is a functor to Hausdorff spaces and open embeddings, and is called ``take the zero set of the section $f$." The second is a functor to smooth manifolds, called ``forget the vector bundle $E$ and the section $f$."}
\end{remark}

\begin{prop}\label{weak-pullback-charts}
The category $\cT$ has weak pushouts, and hence $\CoSpan^\bullet(\cT)$ is a Kan complex. \jake{[This is not true, only locally true.]}
\end{prop}

\begin{proof}
\jake{[To be done.]}
\end{proof}

\begin{prop}\label{weak-initial-charts}
Let $\cU$ be a thickening chart, and let $\cT_\cU$ denote the full subcategory of $\cT$ on those thickening charts $\cV$ such that $\cU$ and $\cV$ are in the same connected component of $\CoSpan^\bullet(\cT)$. Then $\cT_\cU$ has a weak initial object, and hence $\CoSpan^\bullet(\cT)$ is the disjoint union of contractible components. \jake{[This is not true, only locally true.]}
\end{prop}

\begin{proof}
\jake{[To be done.]}
\end{proof}

\subsection{Model structure on thickening charts}

\begin{definition}\label{thickening-model-structure}
We define a category of thickening charts and their tangent complexes, together with specified classes of cofibrations, fibrations, and weak equivalences, as follows.
\begin{itemize}

\item Objects are thickening charts $\cX = (Y_\cX, E_\cX, f_\cX)$, consisting of a smooth manifold $Y$, a smooth vector bundle $E_\cX$ over $Y_\cX$, and a smooth section $f_\cX$ of $E_\cX$. We will use the notation $\cX_0 = f_\cX^{-1}(0) \subset Y_\cX$ to denote the underlying topological space of $\cX$.

\item A pre-morphism $\cF = (U_\cF, i_\cF, j_\cF): \cX \to \cZ$ is given by an open neighborhood $U_\cF$ of $\cX_0$ in $Y_\cX$, together with a smooth map $i_\cF: U_\cF \to Y_\cZ$ and a smooth map $j_\cF: \rest{E_\cX}{U_\cF} \to \rest{i_\cF^* E_\cZ}U_{\cF}$ of vector bundles such that $j_\cF \circ f_\cX = i_\cF^*(f_\cZ)$. A morphism is an equivalence class of pre-morphisms, where two pre-morphisms defined on neighborhoods $U_1, U_2$ are declared equivalent if they agree on a smaller neighborhood $U \subset U_1, U_2$. We will usually abuse notation and consider $\cF$ above to be a morphism, remembering that it is just the germ of $\cF$ around $\cX_0$ that matters.

\item Given a thickening chart $\cX$ and a point $x \in \cX_0$, we define the tangent complex $T_x \cX$ of $\cX$ at $x$ to be the chain complex
\[\xymatrix{ \cdots & 0 \ar[l] & \left(E_\cX\right)_x \ar[l] & T_x Y_\cX \ar[l]_-{(df_\cX)_x} & 0 \ar[l] & \cdots \ar[l] }\]
with $T_x Y_\cX$ in degree zero. This is well defined because $f_\cX(x) = 0$. The assignment $(\cX, x) \mapsto T_x \cX$ defines a functor from pointed thickening charts to chain complexes. We will use $d \cF_x: T_x \cX \to T_{i_\cF(x)} \cZ$ to denote the action of this functor on a morphism $\cF: \cX \to \cY$.

\item A morphism $\cF : \cX \to \cZ$ is a cofibration if $d \cF_x$ is a cofibration of chain complexes, i.e. a level-wise injection, for every $x \in \cX_0$.

\item A morphism $\cF : \cX \to \cZ$ is a fibration if $d \cF_x$ is a fibration of chain complexes, i.e. a level-wise surjection, for every $x \in \cX_0$.

\item A morphism $\cF : \cX \to \cZ$ is a weak equivalence if the induced map $\rest{i_\cF}{\cX_0}: \cX_0 \to \cZ_0$ is a homeomorphism, and if $d \cF_x$ is a quasi-isomorphism of chain complexes for every $x \in \cX_0$.

\end{itemize}
\end{definition}

\begin{prop}\label{model-structure}
The above classes of cofibrations, fibrations, and weak equivalences is a model structure on the category of thickening charts, in the sense of Goerss-Jardine. \jake{[Add reference.]} \jake{Note that the category of thickening charts is not complete or cocomplete.}
\end{prop}

\begin{proof}

First, we must show that the weak equivalences satisfy the 2-out-of-3 property, but this is immediate from the same fact about homeomorphisms of topological spaces and quasi-isomorphisms of chain complexes. Similarly, the closure of cofibrations, fibrations, and weak equivalences under retracts follows from the corresponding facts about embeddings, fiber bundles, and homeomorphisms \jake{[Check this!]}, as well as for cofibrations, fibrations, and quasi-isomorphisms of chain complexes.

To show the existence of factorizations, let $\cF: \cX \to \cZ$ be any morphism. WLOG, we may assume that $U_\cF = Y_\cX$ \jake{[Add lemma explaining that open inclusions arke isos]}. Consider the graph $\Gamma \subset Y_\cX \times Y_\cZ$ of the map $i_\cF$. By the tubular neighborhood theorem, there exists an open neighborhood $U$ of $\Gamma$ such that $U \cong \Gamma \times $

\end{proof}

\subsection{Simplicial sheaf of $n$-thickenings}

\begin{defn}\label{n-thickening}
Let $X$ be a Hausdorff space. An {\bf $n$-thickening} is the data of a smooth manifold $Y$, together with vector spaces $E_0, \dots, E_n$ \jake{[maybe to avoid needing to sheafify, let $E_i$ be a vector bundle over $Y$]} and maps $\sigma_i: Y \to E_i$, and a homeomorphism $\psi: X \to \sigma^{-1}(0)$ (usually suppressed), where $\sigma = \sigma_0 \oplus \cdots \oplus \sigma_n$. We also demand a transversality requirement for $n \geq 1$. \jake{[Spell this out... it should be that all the simultaneous zero sets except the zero set of all the $\sigma_i$ at once are smooth manifolds cut out transversely.]}
\hiro{We certainly should spell this out; your transversality condition sounds like it only kicks in for $k$-simplices with $k \geq 2$.}

Two $n$-thickenings $(Y^k, (E_i^k, \sigma_i^k))$, for $k = 1, 2$, are declared equivalent if they are isomorphic when restricted to open neighborhoods $X \subset U^k \subset Y^k$ (thus $n$-thickenings only depend on the germs of the functions $\sigma_i$ around $X$).
\hiro{It might be nicer to talk about refinements. There's a filtered structure on these $n$-thickenings given by refinements---a refinement would be an (open?) embedding of the manifolds, together with embeddings of vector spaces that are all compatible with the $s_i$. There's probably a clean categorical way to say this---there's a category of $n$-thickenings that has a forgetful functor to $\Delta^{\op}$.}
\end{defn}

\jake{[Details here to be worked out and clarified. In particular, need to explain what ``compatible with $\cA$" means, as well as addressing the fact that the zero sets $f_i^{-1}(0)$ won't be smooth manifolds. Proving the simplicial identities would be good too.]}

Let $X$ be a Hausdorff space. There is a simplicial (pre?)sheaf $\cT \cH_X^\bullet$ on $X$, called the {\bf thickening sheaf} of $X$, given on an open $U \subset X$ by
\[ \cT \cH_X^n(U) = \left\{ \text{$n$-thickenings of $U$} \right\}. \]
The face maps are given by
\[ d_i : \left( Y, (E_0, \sigma_0; \dots; E_n, \sigma_n) \right) \mapsto \left( \sigma_i^{-1}(0), (E_0, \sigma_0; \dots; \wh E_i; \wh \sigma_i, \dots ; E_n, \sigma_n) \right), \]
and the degeneracies are given by
\[ s_i : \left( Y, (E_0, \sigma_0; \dots; E_n, \sigma_n) \right) \mapsto \left( Y \times E_i, (E_0, \sigma_0; \dots; E_i, \pi_2 - \sigma_i \circ \pi_1; E_i, \pi_2; \dots; E_n, \sigma_n) \right). \]
Finally, if $V \subset U$ is open, then the restriction map is given by
\[ r_{U, V}: \left( Y, (E_0, \sigma_0; \dots; E_n, \sigma_n) \right) \mapsto \left( W, (E_0, \rest{\sigma_0}{W}; \dots; E_n, \rest{\sigma_n}{W}) \right), \]
where $W \subset Y$ is any open subset with $W \cap U = V$. Any two choices of such a $W$ are equivalent, since we are working locally around $X$.

\begin{prop}\label{thickenings-are-kan-complexes}
The simplicial set $\cT \cH^\bullet(X)$ is a Kan complex.
\end{prop}

\begin{proof}
\jake{[Proof only works locally on $X$ really, but sheafifying should solve that.]} By symmetry in the definition of $\cT \cH^\bullet(X)$, it suffices to prove the claim for a $0$-horn, given by the $n$-thickenings
\[ (Y^k, (E_0, \sigma_0^k; \dots; \wh E_k, \wh \sigma_k^k; \dots; E_n, \sigma_n^k)), \]
for $1 \leq k \leq n$. Define
\[ Z^k = (\sigma_1^k)^{-1}(0) \cap \dots \cap \wh {(\sigma_k^k)^{-1}(0)} \cap \dots \cap (\sigma_n^k)^{-1}(0) \subset Y^k. \]
We have that $Z^k$ is a smooth submanifold of $Y^k$, and further we have $Z^1 \cong \dots \cong Z^n$ because the given data is a horn; we now suppress these isomorphisms and write $Z$ for this space. Now, define
\[ W^{k, j} = (\sigma_1^k)^{-1}(0) \cap \dots \cap \dots \cap \wh {(\sigma_j^k)^{-1}(0)} \cap \dots \cap \wh {(\sigma_k^k)^{-1}(0)} \cap \dots \cap (\sigma_n^k)^{-1}(0) \subset Y^k, \]
for $1 \leq j \neq k \leq n$. We have that $W^{k, j}$ is a smooth submanifold of $Y^k$, and further we have that $W^{1, j} \cong \dots \cong \wh {W^{j, j}} \cong \dots W^{n, j}$, which we will suppress and denote by $W^j$.

By the inverse function theorem, the transversality conditions in $\cT \cH^\bullet(X)$, and the fact that we are identifying $n$-thickenings which agree in a neighborhood of $X$, we may assume WLOG, but not canonically, that $W^j \cong Z \times E_j$ \jake{[locally on X]}, and further that
\[ Y^k = Z \times E_1 \times \dots \times \wh E_k \times \dots \times E_n, \]
in such a way that $\sigma_i^k = \pi_{E_i}$. But we may then define an $(n + 1)$-thickening that fills the horn, namely
\[ \left( Z \times E_1 \times \dots \times E_n, \left( E_0, \sigma_0; E_1, \pi_{E_1}; \dots; E_n, \pi_{E_n} \right) \right), \]
where we have defined
\[\sigma_0(z, x_1, \dots, x_n) = \sum_{\varnothing \neq I \subseteq \{1, \dots, n\}} (-1)^{\abs{I}} \sigma_0^I(z, x_1, \dots, \wh x_I, \dots, x_n), \]
where $\sigma_0^I: Z \times E_1 \times \dots \times \wh E_I \times \dots \times E_n \to E_0$ is the restriction of $\sigma_0^i$ for any $i \in I$. It is easy to check that we have
\[ \sigma_0(z, x_1, \dots, x_i = 0, \dots, x_n) = \sigma_0^i(z, x_i, \dots, \wh x_i, \dots, x_n), \]
as well as that the transversality condition is satisfied, and thus this is a filler for the given horn.
\end{proof}

\begin{prop}\label{thickenings-are-discrete}
The simplicial set $\cT \cH^\bullet(U)$ is equivalent to a discrete simplicial set.
\end{prop}

\begin{proof}
\jake{[Proof again only works locally on $X$, some sheafy details to be worked out.]}

We show that each component of $\TH^\bullet(U)$ is contractible. Let $p = (Y, (E, f)) \in \TH^0(U)$ be such that $Y$ is of minimal dimension among all thickenings in the connected component of $p$. We claim that any $n$-simplex $(W, (E_0, f_0, \dots, E_n, f_n))$ with vertices at $p$ is necessarily degenerate at $p$, which would imply that the component of $p$ is contractible since $X$ is a Kan complex.

We proceed by induction on $n$. For $n = 1$, suppose we have a $1$-simplex $q = (W, (E_0, f_0, E_1, f_1))$ with vertices given by $p$. Explicitly, this means we are given embeddings $\iota_i : Y \to W$ such that $Y_i = \im(\iota_i)$ satisfies $Y_0 = f_1^{-1}(0)$ and vice versa, together with (not necessarily linear) isomorphisms of bundles $u_i: \overline{E} \to \iota_i^* \overline{E_i}$ such that $u_i \circ f = \iota_i^* (f_i)$. Choose $x \in U$, and consider the map
\[ df_0 \oplus df_1 : T_x W \to E_0 \oplus E_1. \]
We claim that the rank of this map is equal to the dimension of $E \cong E_i$. Suppose not. Then we would have that $\ker(df_0 \oplus df_1)$ had dimension strictly lower than that of $\ker(df_i)$, which by Lemma \ref{minimal-dimension} would construct a thickening of strictly lower dimension than $Y$ in the same connected component, contradicting the minimality of $Y$.

Now, we have that $F = \im(df_0 \oplus df_1)$ has dimension equal to that of $E_i$, and further that $\pi_i : F \to E_i$ is an isomorphism since $df_i$ is surjective. Thus, we may identify $E_i$ with $F$ by this given isomorphism, and we may assume that
\[ q = (W, (F, f_0, F, f_1)), \]
such that $\im(df_0 \oplus df_1)$ at $x$ is the diagonal $F \oplus F$, or equivalently, such that $df_0 = df_1$ as maps $T_x W \to F$. Now, choose an isomorphism $h: E \to F$ of vector spaces, and consider the map

\end{proof}

\begin{lemma}\label{minimal-dimension}
Let $(W, (E_0, f_0, E_1, f_1))$ be a $1$-thickening of $U$ with endpoints $p_i = (Y_i, (E_i, f_i))$, for $Y_0 = f_1^{-1}(0)$ and vice versa. Let $x \in U$, and let
\[d = \dim(T_x Y_1 \cap T_x Y_2).\]
Then there exists a $0$-thickening $q = (Z, (H, h))$ of some neighborhood $V$ of $x$ in $U$ with $\dim(Z) = d$, and such that $q$ is in the same connected component of $\TH^\bullet(V)$ as $r_{U, V}(p_i)$.
\end{lemma}

\begin{proof}

Consider
\[ df_1 \oplus df_2 : T_x W \to E_1 \oplus E_2, \]
and let $F = \im(df_1 \oplus df_2)$. Choose a complement $H$ for $F$ such that $H \subset E_1$; such an $F$ exists because $df_2$ is surjective by transversality. Let $\pi_F, \pi_H$ be the projections induced by this decomposition, and consider the function 
\[g = \pi_F \circ (f_1 \oplus f_2) : W \to F.\]
We have that $dg$ is surjective at $x$, and so there is a neighborhood $V$ of $x$ where $Z = g^{-1}(0)$ is a smooth manifold. Consider now the $0$-thickening $q$ of $V$ given by
\[ q = (Z, (H, h = f_1)),\]
which is well defined, since on $Z$ we have that $f_1$ takes values in $H$. This thickening has
\[ \dim(Z) = \dim(\ker(df_2 \oplus df_2)) = d,\]
and further there is a $1$-thickening
\[ (f_1^{-1}(H), (H, f_1, E_2, f_2)), \]
with endpoints given by
\[ (f_1^{-1}(0), (E_2, f_2)) = (Y, (E, f)), \]
and
\[ (f_1^{-1}(H) \cap f_2^{-1}(0), (H, f_1)) = (Z, (H, h)), \]
and thus we have found the desired thickening.

\end{proof}

\begin{definition}\label{implicit-manifold}
An {\bf implicit manifold} is a compact Hausdorff space $X$ together with a choice of global section of $\cT \cH^\bullet(X)$. \jake{This should agree with Pardon's definition of an implicit atlas.}
\end{definition}

We have the following candidate for the structure sheaf of $(X, \cA)$ as a derived manifold. Consider the simplicial sheaf $\cO_X^\bullet$ on $X$, given by
\[ \cO_X^n(U) = \coprod_{(Y, (E_i, \sigma_i)) \in \cT \cH_\cA^n(U)} C^\oo(Y). \]

\hiro{I think this is a beautiful candidate. Do you want to try and work out the speculation? Spivak at some point must do something very similar in his thesis.}

\begin{speculation}
If $X$ is a smooth manifold, then $\cO_X^\bullet(X)$ is equivalent to the discrete simplicial set $C^\oo(X)$. Further, if $X$ is the intersection of the origin in $\RR$ with itself, then $\pi_0(\cO_X^\bullet(X)) \cong \RR$ and $\pi_1(\cO_X^\bullet(X)) \cong \RR$. To see this last part, consider smooth functions on $\RR^n$ that vanish on the coordinate hyperplanes, and see how strong a zero they must have when restricted to another generic plane.
\end{speculation}
%!TEX root = mcnamara-tanaka.tex


\section{Notes from June 22}

Let $X \to S$ be a map of simplicial sets. Of course, it need not be a right fibration. Jake suspects that we can make the a replacement $X \to X' \to S$ where $X \to X$, while the map $X' \to S$ becomes a right fibration. He believes that this replacement is just a cospan construction, where objects are the same objects of $X$, and an edge from $A \to B$ is given by a cospan $A \to A' \leftarrow B$ where $A' \leftarrow B$ is an equivalence. The relevant section in HTT appears to be 5.1, and the proof of Proposition 5.1.1.1 there. It would be great to prove that this cospan construction for thickenings is just a general construction, and an explicit model of a fibrant replacement.

Another issue that we should make an immediate goal is how to find fibrant replacements in the category of $C^\infty$ rings---i.e., simplicial, product-preserving functors out of the category of manifolds. This is important to us to understand the ``fat point'' example of Spivak. For instance, let $G$ be the $C^\infty$ ring given by the sheaf on the fat point. The fibrant replacement of this in the category of $C^\infty$ rings on a point is {\em not} just fibrantly replacing (as a simlicial set) $G(M)$ for every manifold $M$---for then this would not be a $C^\infty$ ring (it may not preserve products). We should really figure out the fibrant replacement to try and prove that this replacement is equivalent to Jake's intuition/construction of the fat point.

Yi-Fei and Jake showed that $pt \cap pt \cap pt$ in $\RR$ has non-trivial $\pi_2$. But it also seems like we might be doing things that are only 1-derived; and Jake things cotangent complexes only have $H^{-1}$ and no lower cohomology. This seems like an issue. 

It's also unclear to what extent Pardon's implicit atlases are equivalent to ours, since he only imposes regularity at the very end.

Also, we wondered to what extent model categories need all finite (co)limits. Hiro advocated that---before digging through the literature on all the model category proofs requiring the existence of (co)limits---we just try to write down an equivalence (of $\infty$-categories, or of simplicially enriched categories) between Jake's category of thickenings and Spivak's category. In the course of writing that down, it should become clearer what we properties we really need.


\subsection{\jake{Follow-up thoughts}}

\begin{enumerate}

\item The thing I presented as a model structure is definitely not. An example of a map that has no replacement by a fibration is the unique map from the point to the derived loop space of $\RR$ (the thickened point we've been talking about). This is essentially because any replacement of the map from the zero complex to the chain complex with $\RR$ in degree $-1$ with a fibration necessarily has something in degree $-2$, which is impossible. One consequence is that the derived double loop space of $\RR$ doesn't exist in this category.

\item Spivak's framework is very definitely only talking about $1$-derived manifolds, and not $\infty$-derived manifolds. In particular, not all pullbacks exist, but only pullbacks with the base a manifold. This is related to the above point; I think that any map with target a manifold can be replaced by a fibration.

\item I think Pardon's work is also really only dealing with $1$-derived manifolds, and the fact that he has to use charts that aren't smooth everywhere is only for convenience.

\item I'm pretty sure that there is a canonical structure of an implicit manifold on the zero locus of a Fredholm section in the sc-calculus picture as well, and it might be interesting to see how all these things interact.

\end{enumerate}
